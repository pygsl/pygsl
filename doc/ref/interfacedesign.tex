\chapter{Design of the \pygsl{} interface}

The GSL library was implemented using the C language. This implies that 
each function uses a certain type for the different variables and are fixed
 to one specific type. The wrapper will try to convert each argument to the approbriate
C type. 
The \pygsl{} interface
tries to follow it as much as possible but only as far as resonable. 
For example the definition of the poly_eval function in C is given by
\begin{funcdesc}{\texttt{double} gsl_poly_eval}
                {\texttt{const double} C[], \texttt{const int} LEN, \texttt{const double} X}
\end{funcdesc}

The corresponding python wrapper was implemented by
\begin{funcdesc}{poly.poly_eval}{C, x}
\end{funcdesc}
as the wrapper can get the length of any python object and then fill the len variable. 
The mathematical calculation is performed by the GSL library. Thus the calculation is limited 
to the precision provided by the underlying hardware.

Default arguments are used to allocate workspaces on the fly if not provided by the user. 
Consider for example the fft module. The function for the real forward transform is
named 

\begin{funcdesc}{\texttt{int} gsl_fft_real_transform}
{\texttt{double DATA[]}, 
 \texttt{size_t STRIDE},
 \texttt{size_t N}, 
 \texttt{const gsl_fft_real_wavetable * WAVETABLE},
 \texttt{gsl_fft_real_workspace * WORK}
}
\end{funcdesc}

The corresponding python wrapper is found in the fft module called
real_transform
\begin{funcdesc}{real_transform}
{data, \optional{space, 
    table, 
    output}}
\end{funcdesc}
The wrapper will get the stride and size information from the data object provided
by the user. If space or table are not provided, these objects will be generated on 
the fly. As the GSL function applies the transformation in space, an internal copy is 
made of the data and only then the object is passed to the \gsl{} function. If an output
object is provided the data will be copied there instead. \pygsl{} will always make copies
of objects which would be otherwise modified in place.

\section{Callbacks}

Solvers require as one argument a user function to work on which have to be provided by the
user. These callbacks typically are of the form
\begin{funcdesc}{f}{x, params}
  \dots\\
  return result
\end{funcdesc}
Please note that this function must return the exact number of arguments
as given in the example. The wrappers around callbacks go a long way to try to provide
meaningfull error messages. If a solver fails, please check that the number of input and 
output arguments it takes are correct

\section{Error handling}
As GSL is a C library error handling is implemented using an error handler and return values.
\pygsl{} generates python exceptions out of these values. See \module{pygsl.errors} 
(chapter~\ref{cha:error-module}) for a list of the exceptions.

\section{The documentation gap}

\pygsl{} does still lack an approbriate documentation. Most documentation is accessible over
the internal documentation strings. These are accessible as \code{__doc__} attributes (the help
function does not always show them).  It can be sometimes necessary to create an 
object to see its methods as well as the documentation of the methods
 (e.g.a random number generator in the rng module to see its methods). 
The \file{example} directory contains examples for (nearly each) module.

Please feel welcome to add to the documentation!

