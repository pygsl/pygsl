\section{Status}

\paragraph*{Status of GSL-Library}
The gsl-library is since version 1.0 stable and for general use.
More information about it at \url{http://sources.redhat.com/gsl/}.

\paragraph*{Status of this interface}
Nearly all modules are wrapped. The sum and hankel modules have been
wrapped. But noone responded to the request of support (providing an example,
test the code for proper functioning)
The code is quite stable now XXX 

\paragraph*{Retriving the Interface}
You can download it here: \url{http://sourceforge.net/projects/pygsl}

\section{Requirements}

To build the interface, you will need
\begin{itemize}
\item \ulink{gsl-1.x}{http://sources.redhat.com/gsl},
\item \ulink{python2.2}{http://www.python.org} or better,
\item \ulink{NumPy or numarray}{http://numpy.sf.net}, and
\item a c compiler (like \ulink{gcc}{http://gcc.gnu.org}).
\end{itemize}

Supported Platforms are:
\begin{itemize}
\item Linux (Redhat/Debian/SuSE) with python2.* and gsl-1.*
\item SUN
\item Cygwin
\item Win32
\item MacOS X
\end{itemize}

\section{Installing GSL interface}

\program{gsl-config} must be on your path:\nopagebreak
\begin{verbatim}
# unpack the source distribution
gzip -d -c pygsl-x.y.z.tar.gz|tar xvf-
cd pygsl-x.y.z
# do this with your prefered python version
# to set the gsl location explicitly use setup.py --gsl-prefix=/path/to/gsl
python setup.py build
# change to an user id, that is allowed to do installation
python setup.py install
\end{verbatim}
Ready....

{\bf Do not test the interface in the distribution root or in the directories \file{src} or \file{pygsl}.}

\paragraph*{Uninstall GSL interface}
\code{rm -r }"python install path"\code{/lib/python}"version"\code{/site-packages/pygsl}

\paragraph*{Testing}
the directory \file{tests} will contains several testsuites, based on python \module{unittest}.
The script \file{run_test.py} in this directory will run one after the other.

\paragraph*{Support}
Please send mails to our mailinglist at \email{pygsl-discuss@lists.sourceforge.net}.

\paragraph*{Developement}
You can browse our cvs tree at \url{http://cvs.sourceforge.net/cgi-bin/viewcvs.cgi/pygsl/pygsl/}.
\\
Type this to check out the actual version:
\begin{verbatim}
cvs -d:pserver:anonymous@cvs.pygsl.sourceforge.net:/cvsroot/pygsl login
#Hit return for no password.
cvs -z3 -d:pserver:anonymous@cvs.pygsl.sourceforge.net:/cvsroot/pygsl co pygsl
\end{verbatim}
The script \program{tools/extract_tool.py} generates most of the special function code.

\paragraph*{ToDo}
Implement other parts:


\paragraph*{History}
\begin{itemize}
\item a gsl-interface for python was needed for a project at
\ulink{Center for Applied Informatics Cologne}{http://www.zaik.uni-koeln.de/AFS}.
\item \file{gsl-0.0.3} was released at May 23, 2001
\item \file{gsl-0.0.4} was released at January 8, 2002
\item \file{gsl-0.0.5} is growing since January, 2002
\item \file{gsl-0.2.0} was released at 
\item \file{gsl-0.3.0} was released at 
\item \file{gsl-0.3.1} was released at 
\item \file{gsl-0.3.2} was released at 
\end{itemize}

\paragraph*{Thanks}
Jochen K\"upper (\email{jochen@jochen-kuepper.de}) for \module{pygsl.statistics} part\\
Fabian Jakobs for \module{pygsl.blas}\module{pygsl.eigen}
\module{pygsl.linalg} \module{pygsl.permutation}\\ 
Leonardo Milano for rpm build\\
Eric Gurrola and  Peter Stoltz for testing and supporting the port of pygsl to
the MAC\\
Sebastien Maret for supporting the Fink \url{http://fink.sourceforge.net} port of
pygsl.


\paragraph*{Maintainers}
Achim G\"adke (\email{AchimGaedke@users.sourceforge.net}),
Pierre Schnizer (\email{schnizer@users.sourceforge.net})
