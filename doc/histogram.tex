\declaremodule{extension}{pygsl.histogram}
\moduleauthor{Achim G\"adke}{achimgaedke@users.sourceforge.net}
% for indexing like \refihistogramindex{histogram}{get}
\newcommand{\refhistogramindex}[2]{\texttt{#2}\index{#2@\texttt{#2} (in #1)}}

This chapter is about the \class{histogram} and \class{histogram2d} type that
are contained in \module{pygsl.histogram}.

\section{\protect\class{histogram} --- 1-dimensional histograms}

\index{histogram@\class{histogram} (in pygsl.histogram)}
This type implements all methods on \ctype{struct gsl_histogram}.
These are:

\newcommand{\methodhistogramindex}[1]{\refhistogramindex{histogram}{#1}}
\begin{longtableiii}{l|l|l}{methodhistogramindex}{Method}{Description}{Return value}
\lineiii{alloc}{allocate necessary space}{\texttt{None}}
\lineiii{set_ranges_uniform}{set the ranges to uniform distance}{\texttt{None}}
\lineiii{reset}{sets all bin values to 0}{\texttt{None}}
\lineiii{increment}{increments corresponding bin}{\texttt{None}}
\lineiii{accumulate}{adds the weight to corresponding bin}{\texttt{None}}
\lineiii{max}{returns upper range}{\texttt{float}}
\lineiii{min}{returns lower range}{\texttt{float}}
\lineiii{bins}{returns number of bins}{\texttt{long}}
\lineiii{get}{gets value of indexed bin}{\texttt{float}}
\lineiii{get_range}{gets upper and lower range of indexed bin}{\texttt{(float,float)}}
\lineiii{find}{finds index of corresponding bin}{\texttt{long}}
\lineiii{max_val}{returns maximal bin value}{\texttt{float}}
\lineiii{max_bin}{returns bin index with maximal value}{\texttt{long}}
\lineiii{min_val}{returns minimal bin value}{\texttt{float}}
\lineiii{min_bin}{returns bin index with minimal value}{\texttt{long}}
\lineiii{mean}{returns mean of histogram}{\texttt{float}}
\lineiii{sigma}{returns std deviation of histogram}{\texttt{float}}
\lineiii{sum}{returns sum of bin values}{\texttt{float}}
\lineiii{set_ranges}{sets range according given sequence}{\texttt{None}}
\lineiii{shift}{shifts the contents of the bins by the given offset}{\texttt{None}}
\lineiii{scale}{multiplies the contents of the bins by the given scale}{\texttt{None}}
\lineiii{equal_bins_p}{true if the all of the individual bin ranges are identical}{\texttt{int}}
\lineiii{add}{adds the contents of the bins}{\texttt{None}}
\lineiii{sub}{substracts the contents of the bins}{\texttt{None}}
\lineiii{mul}{multiplicates the contents of the bins}{\texttt{None}}
\lineiii{div}{divides the contents of the bins}{\texttt{None}}
\lineiii{clone}{returns a new copy of this histogram}{\texttt{histogram}}
\lineiii{copy}{copies the given histogram to myself}{\texttt{None}}
\lineiii{read}{reads histogram binary data from file}{\texttt{None}}
\lineiii{write}{writes histogram binary data to file}{\texttt{None}}
\lineiii{scanf}{reads histogram data from file using scanf}{\texttt{None}}
\lineiii{printf}{writes histogram data to file using printf}{\texttt{None}}
\end{longtableiii}

Some mapping operations are supported too:\nopagebreak
\begin{tableii}{l|l}{texttt}{Mapping syntax}{Effect}
\lineii{histogram[index]}{returns the value of the indexed bin}
\lineii{histogram[index]=value}{sets the value of the indexed bin}
\lineii{len(histogram)}{returns the length of the histogram}
\end{tableii}

\begin{seealso}
For the special meaning and details please consult the GNU Scientific Library
reference.
\end{seealso}


\section{\protect\class{histogram2d} --- 2-dimensional histograms}
\index{histogram2d@\class{histogram2d} (in pygsl.histogram)}

Most of the methods are the same as of \class{histogram}:

\newcommand{\methodhistogramddindex}[1]{\refhistogramindex{histogram2d}{#1}}

\begin{longtableiii}{l|l|l}{methodhistogramddindex}{Method}{Description}{Return value}
\lineiii{alloc}{allocate necessary space}{\texttt{None}}
\lineiii{set_ranges_uniform}{set the ranges to uniform distance}{\texttt{None}}
\lineiii{reset}{sets all bin values to 0}{\texttt{None}}
\lineiii{increment}{increments corresponding bin}{\texttt{None}}
\lineiii{accumulate}{adds the weight to corresponding bin}{\texttt{None}}
\lineiii{xmax}{returns upper x range}{\texttt{float}}
\lineiii{xmin}{returns lower x range}{\texttt{float}}
\lineiii{ymax}{returns upper y range}{\texttt{float}}
\lineiii{ymin}{returns lower y range}{\texttt{float}}
\lineiii{nx}{returns number of x bins}{\texttt{long}}
\lineiii{ny}{returns number of y bins}{\texttt{long}}
\lineiii{get}{gets value of indexed bin}{\texttt{float}}
\lineiii{get_xrange}{gets upper and lower x range of indexed bin}{\texttt{(float,float)}}
\lineiii{get_yrange}{gets upper and lower y range of indexed bin}{\texttt{(float,float)}}
\lineiii{find}{finds index pair of corresponding value pair}{\texttt{(long,long)}}
\lineiii{max_val}{returns maximal bin value}{\texttt{float}}
\lineiii{max_bin}{returns bin index with maximal value}{\texttt{long}}
\lineiii{min_val}{returns minimal bin value}{\texttt{float}}
\lineiii{min_bin}{returns bin index with minimal value}{\texttt{long}}
\lineiii{sum}{returns sum of bin values}{\texttt{float}}
\lineiii{xmean}{returns x mean of histogram}{\texttt{float}}
\lineiii{xsigma}{returns x std deviation of histogram}{\texttt{float}}
\lineiii{ymean}{returns y mean of histogram}{\texttt{float}}
\lineiii{ysigma}{returns y std deviation of histogram}{\texttt{float}}
\lineiii{cov}{returns covariance of histogram}{\texttt{float}}
\lineiii{set_ranges}{set the ranges according to given sequences}{\texttt{None}}
\lineiii{shift}{shifts the contents of the bins by the given offset}{\texttt{None}}
\lineiii{scale}{multiplies the contents of the bins by the given scale}{\texttt{None}}
\lineiii{equal_bins_p}{true if the all of the individual bin ranges are identical}{\texttt{int}}
\lineiii{add}{adds the contents of the bins}{\texttt{None}}
\lineiii{sub}{substracts the contents of the bins}{\texttt{None}}
\lineiii{mul}{multiplicates the contents of the bins}{\texttt{None}}
\lineiii{div}{divides the contents of the bins}{\texttt{None}}
\lineiii{clone}{returns a new copy of this histogram}{\texttt{histogram2d}}
\lineiii{copy}{copies the given histogram to myself}{\texttt{None}}
\lineiii{read}{reads histogram binary data from file}{\texttt{None}}
\lineiii{write}{writes histogram binary data to file}{\texttt{None}}
\lineiii{scanf}{reads histogram data from file using scanf}{\texttt{None}}
\lineiii{printf}{writes histogram data to file using printf}{\texttt{None}}
\end{longtableiii}

Some mapping operations are supported too:\nopagebreak
\begin{tableii}{l|l}{code}{Mapping syntax}{Effect}
\lineii{histogram[x\_index,y\_index]}{returns the value of the indexed bin}
\lineii{histogram[x\_index,y\_index]=value}{sets the value of the indexed bin}
\lineii{len(histogram)}{returns the size of the histogram, i.e nx$\times$ny}
\end{tableii}


\begin{seealso}
For the special meaning and details please consult the GNU Scientific Library
reference.
\end{seealso}

\section{\protect\class{histogram_pdf} and \protect\class{histogram2d_pdf}}

To be implemented\dots

%% Local Variables:
%% mode: LaTeX
%% mode: auto-fill
%% fill-column: 79
%% indent-tabs-mode: nil
%% ispell-dictionary: "american"
%% reftex-fref-is-default: nil
%% TeX-auto-save: t
%% TeX-command-default: "pdfeLaTeX"
%% TeX-master: "pygsl"
%% TeX-parse-self: t
%% End:
