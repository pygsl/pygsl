\declaremodule{extension}{pygsl.statistics}
\moduleauthor{Jochen K\"upper}{jochen@jochen-kuepper.de}
\index{mean}
\index{standard deviation}
\index{variance}
\index{estimated standard deviation}
\index{estimated variance}
\index{t-test}
\index{range}
\index{min}
\index{max}

This chapter describes the statistical functions in the library.  The
basic statistical functions include routines to compute the mean,
variance and standard deviation. More advanced functions allow you to
calculate absolute deviations, skewness, and kurtosis as well as the
median and arbitrary percentiles.  The algorithms use recurrence
relations to compute average quantities in a stable way, without large
intermediate values that might overflow. 

All functions work on any Python sequence (of the appropriate
data-type), but see section \ref{Speed considerations} for advantages and
drawbacks of different kinds of input data.


\section{Organization of the module}

The parts of the GSL functions names, providing artificial name spaces,
are mapped to modules and submodules in pygsl.  That is
\code{gsl_stats_mean} can be found as \code{pygsl.statistics.mean} and
\code{gsl_stats_long_mean} as \code{pygsl.statistics.long.mean}.

The functions in the module are available in versions for datasets in
the standard floating-point and integer types. The generic versions
available in the \code{pygsl.statistics} module are using the generic
GSL \code{double} versions.  The submodules use GSL functions according
to the submodule name, e.g. long for \code{pygsl.statistics.long}.

Currently implemented submodules are \code{pygsl.statistics.double} and
\code{pygsl.statistics.long}.



\section{Speed considerations}
\label{Speed considerations}
All functions work on any Python sequence type but are optimized for use
with NumPy arrays. It is strongly suggested that you install NumPy
(available from \url{http://www.numpy.org})!

If you pass NumPy arrays of the \emph{correct data-type} as input data
to any of the functions they are passed straight to the C functions
along with the stride information of the data.

If you pass generic (non-NumPy) Python sequences or NumPy arrays of the
wrong data-type a suitable copy of the data will be created and passed
to the function.


\section{Further Reading}

See the gsl reference manual for a description of all available
functions and the calculations they perform.
