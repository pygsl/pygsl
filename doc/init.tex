\declaremodule{extension}{pygsl.init}
\moduleauthor{Achim G\"adke}{achimgaedke@users.sourceforge.net}

This module is called the first time when loading \module{pygsl}.
All code and following procedures are called once.

\section{Exception handling}

GSL provides a selectable error handler, that is called for occuring errors (like domain errors, division by zero, etc. ).
\module{pygsl.init} installs a handler by calling \cfunction{gsl_set_error_handler} to set an appropiate exception from \module{pygsl.errors}.
A \module{pygsl} interface function should return \code{NULL} in case of an error, so the exception is raised.
If this handler is called more than once before returning to python, only the first set exception is raised.

Here is a python level example:

\begin{verbatim}
import pygsl.histogram
import pygsl.errors
hist=pygsl.histogram.histogram2d(100,100)
try:
   hist[-1,-1]=0
except pygsl.errors.gsl_Error:
   print "array indices out of range"
\end{verbatim}

\section{IEEE-mode}

The IEEE mode is set from the environment variable \code{GSL_IEEE_MODE} via \cfunction{gsl_ieee_env_setup()}.
After the initialisation use \module{pygsl.ieee} for manipulation.

\section{random number generators}

Also the random number generator can be initialised from the environment variables \code{GSL_RNG_TYPE} and
\code{GSL_RNG_SEED} using the gsl function \cfunction{gsl_rng_env_setup()}.
The random number generators are initialised with \code{GLS_RNG_SEED} and the default generator can be created by \code{my_rng=gsl_rng()}.
